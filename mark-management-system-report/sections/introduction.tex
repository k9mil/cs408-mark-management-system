\section{Introduction}


This is a concise and clear overview of your dissertation(more or less 2-3 pages). You can start off the project description provided of the project that was allocated to you and flesh it out.

Include (1) the problem you have tackled, (2) why this problem is worth addressing, (3) what you did to address it - in broad terms. Detail will come later.



i.e., issue(s) on which the research will focus, shall be clearly identified and described. You shall refer to past research work relevant to the topic and objectives, i.e, of the study. You shall outline where applicable the potential research output with respect to research transfer and uptake by the community.

The introduction says: this is an overview of the project. This is why I did it (the problem) and how I tackled it. It is the \emph{runway} into the project. It lets the marker know what to expect of your report.

If you're doing a research project, this would be the place to include the research questions you plan to address. For example:

\begin{description}
\item [RQ1:] Where did James Bond come from?

\item [RQ2:] Why are pumpkins orange?
\end{description}

If you're doing a project of type 1, include a list of objectives. For example:
\begin{description}
\item [Objective 1:] Provide software to allow James Bond to become invisible.

\item [Objective 2:] Provide software to keep track of all loyalty points in one place.
\end{description}

Provide a `map' of your dissertation. For example: Section \ref{backg} reviews the background literature that was reviewed to inform this project. Then Section \ref{process}.....
