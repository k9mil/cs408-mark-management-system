
\section{Specification \& Design}\label{process}

Describe all details of the  design and procedures used to achieve the project objectives. Do this chronologically. 


It should be detailed enough to allow for an assessment of the rigour of your process, and, in the case of research projects, in terms of how well grounded your research is in the research literature. In these cases, refer back to relevant sections in the previous chapter.

Say which  software lifecycle approach you used e.g., Waterfall, Spiral, Agile. 

How did you gather user requirements?

\subsection{Methodology}
Which methodology did you choose?

\subsection{Analysis}
How did you decide on the particular software artifact you decided to develop?

\subsection{Requirements}
Here you explain what the functional and non-functional requirements are. Explain how you prioritised them. 
  See \url{https://www.nuclino.com/articles/functional-requirements} for more information. 

   \textbf{Functional requirement:} "The system must \textbf{\emph{do}} [requirement]."

    \textbf{Non-functional requirement:} "The system shall \textbf{\emph{be}} [requirement]."


Well-written functional requirements typically have the following characteristics:
\begin{description}



    \item[Necessary:] Although functional requirements may have different priority, every one of them needs to relate to a particular business goal or user requirement.

  \item[Concise:] Use simple and easy-to-understand language without any unnecessary jargon to prevent confusion or misinterpretations.

 \item[Attainable:] All requirements you include need to be realistic within the time and budget constraints set in the business requirements document.

 \item[Granular:] Do not try to combine many requirements within one. The more precise and granular your requirements are, the easier it is to manage them.

 \item[Consistent:] Make sure the requirements do not contradict each other and use consistent terminology.

 \item[Verifiable:] It should be possible to determine whether the requirement has been met at the end of the project.
    \end{description}

\subsubsection{Functional Requirements}
This is the \textbf{WHAT} of your artifact.

Functional requirements are product features that developers must implement to enable the users to achieve their goals. They define the basic system behavior under specific conditions. 

Functional requirements need to be clear, simple, and unambiguous.Examples:
\begin{itemize}
    \item The system must send a confirmation email whenever an order is placed.

      \item The system must allow blog visitors to sign up for the newsletter by leaving their email.

      \item The system must allow users to verify their accounts using their phone number.
\end{itemize}


\subsubsection{Non-Functional Requirements}
This is the \textbf{HOW} of your artifact. Example non-functional requirement: ``When the submit button is pressed, the confirmation screen must load within 2 seconds.''

\subsection{Design}

\subsubsection{Interface Design}
Explain how you used wireframes, and how you tested these to design the user interface.


 \subsubsection{System Design}
 Show how you designed your database (if appropriate) and how you designed your system architecture, and the individual parts. Use UML and an Entity Relationship diagram

